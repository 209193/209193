\documentclass[a4paper,10pt]{scrartcl}

\usepackage{polski}
\usepackage[utf8]{inputenc}
\usepackage{graphicx}
\usepackage{enumerate}
\usepackage{pdflscape}

\title{Laboratorium 7}
\author{Filip Malinowski}
\date{\today}

\pdfinfo{%
  /Title    (Laboratorium 7)
  /Author   (Filip Malinowski)
}

\begin{document}

\title{Sprawozdanie z laboratorium 7}
\author{Filip Malinowski}
\date{\today}

\maketitle

W programie zastosowalem wzorzec obserwatora.
Zdefiniowalem do tego ogolnego obserwatora
Observer oraz ogolnego obserwowanego Subject.
Klasa Benchmark jest obserwowanym.
Kolejne klasy symulujace algorytmy dziedzicza
po klasie Benchmark bedac tym samym obserwowanymi.
Dany algorytm dodaje na liste
obserwatorow obserwatora zapisujacego,
ktory przy kazdym powiadomieniu pobiera
aktualne czasy oraz ilosci danych wykonane
w danym algorytmie. Pobrane dane nastepnie
zapisuje do pliku zachowujac zrozumialy format.

Sortowania zostaly uogolnione do wszystkich
struktur napisanych na tablicy.

Zostaly tez wprowadzone poprawki przy strukturach,
benchmarku i pochodnych w celu zmiejszenia
zuzycia pamieci.

W programie zostaly przy okazji dodane komunikaty
o pracy konkretnych obiektow kierowane na
standardowy strumien bledow.
Kazdy obserwowany oraz obserwator przy tworzeniu
moze otrzymac identyfikator wykorzystywany
przy komunikatach.

Dokumentacja zostala kolejny raz sprawdzona
i uzupelniona. Znaleznione bledy zostaly poprawione.

\end{document}