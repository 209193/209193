\documentclass[a4paper,10pt]{scrartcl}

\usepackage{polski}
\usepackage[utf8]{inputenc}
\usepackage{graphicx}
\usepackage{enumerate}
\usepackage{pdflscape}

\title{Laboratorium 5}
\author{Filip Malinowski}
\date{\today}

\pdfinfo{%
  /Title    (Laboratorium 5)
  /Author   (Filip Malinowski)
}

\begin{document}

\title{Sprawozdanie z laboratorium 5}
\author{Filip Malinowski}
\date{\today}

\maketitle

Do programu zostały dodane lista asocjacyjna
oraz tablica z haszowaniem.

Lista asocjacyjna jest odpowiednikiem tablicy
asocjacyjnej wykonanej na liście. Do listy
dodałem podstawowe metody do wstawiania pary
klucz-wartość, do usuwania wszystkich elementow
z listy, itd.

W tablicy z haszowaniem wybrałem haszowanie
wzorowane na kukułczym. Zastosowałem do tego
dwie tablice przechowujące pary klucz-wartość.
Lewa tablica korzysta z funkcji haszującej
Shift-Add-XOR. Prawa tablica korzysta z
Fowler/Noll/Vo.
Wybrałem ten typ tablicy z haszowaniem
ze względu na teoretycznie wyższą szybkość
działania w porównaniu do wstawiania
progresywnego przy wystąpieniu konfliktu.
Różni się to od haszowania kukułczego tym,
że nie losuję tutaj funkcji haszujących
tylko korzystam z dwóch predefiniowanych.

\end{document}
